
\documentclass[]{svmono}

\usepackage{fancybox}
\usepackage{graphicx}
\usepackage{pdfpages}
\usepackage{color}
\usepackage{epstopdf}
\usepackage[margin=1in, vmargin=1in]{geometry}
\usepackage{amsmath}
\usepackage{float}
\usepackage{listings}
\usepackage{verbatim}
\usepackage{booktabs}
\usepackage{tabularx}
\usepackage{longtable}
\usepackage{amsmath}
\usepackage{caption}

\usepackage{subcaption}


\usepackage{enumerate}

\usepackage{bashful}

\usepackage{multicol}


\usepackage{textcomp}
\usepackage{hyperref}
%\usepackage[hyphenbreaks]{breakurl}
%\usepackage[hyphens]{url}


\sloppy
\definecolor{lightgray}{gray}{0.5}
\setlength{\parindent}{20pt}


\definecolor{mygreen}{rgb}{0,0.6,0}
\definecolor{mygray}{rgb}{0.5,0.5,0.5}
\definecolor{mymauve}{rgb}{0.58,0,0.82}

\lstset{ %
  backgroundcolor=\color{white},   % choose the background color; you must add \usepackage{color} or \usepackage{xcolor}
  basicstyle=\footnotesize,        % the size of the fonts that are used for the code
  breakatwhitespace=false,         % sets if automatic breaks should only happen at whitespace
  breaklines=true,                 % sets automatic line breaking
  captionpos=b,                    % sets the caption-position to bottom
  commentstyle=\color{mygreen},    % comment style
  deletekeywords={...},            % if you want to delete keywords from the given language
  escapeinside={\%*}{*)},          % if you want to add LaTeX within your code
  extendedchars=true,              % lets you use non-ASCII characters; for 8-bits encodings only, does not work with UTF-8
  frame=single,                    % adds a frame around the code
  keepspaces=true,                 % keeps spaces in text, useful for keeping indentation of code (possibly needs columns=flexible)
  keywordstyle=\color{blue},       % keyword style
  language=Python,                 % the language of the code
  morekeywords={*,...},            % if you want to add more keywords to the set
  numbers=left,                    % where to put the line-numbers; possible values are (none, left, right)
  numbersep=5pt,                   % how far the line-numbers are from the code
  numberstyle=\tiny\color{mygray}, % the style that is used for the line-numbers
  rulecolor=\color{black},         % if not set, the frame-color may be changed on line-breaks within not-black text (e.g. comments (green here))
  showspaces=false,                % show spaces everywhere adding particular underscores; it overrides 'showstringspaces'
  showstringspaces=false,          % underline spaces within strings only
  showtabs=false,                  % show tabs within strings adding particular underscores
  stepnumber=2,                    % the step between two line-numbers. If it's 1, each line will be numbered
  stringstyle=\color{mymauve},     % string literal style
  tabsize=2,                       % sets default tabsize to 2 spaces
  title=\lstname                   % show the filename of files included with \lstinputlisting; also try caption instead of title
}



\setlength{\parindent}{10pt}


\begin{document}




\begin{titlepage}    
\begin{center}
\vspace*{0in}
\huge{\sc  Old Dominion Univeristy\\  }



\vspace{1in}
\Large{\sc CS 495: Introduction to Web Science \\ Instructor: Michael L. Nelson, Ph.D \\ Fall 2014 4:20pm - 7:10pm R, ECSB 2120\\}

\vspace{1in}
\Large{Assignment \# 8\\}

\vspace{.5cm}
\Large{ \sc George C.  Micros  UIN: 00757376\\ }



\vspace {7cm}

{\large \bf {Honor Pledge}}\\
{I pledge to support the Honor System of Old Dominion University. I will refrain from any form of academic dishonesty or deception, such as cheating or plagIiarism. I am aware that as a member of the academic community it is my responsibility to turn in all suspected violations of the Honor Code. I will report to a hearing if summoned. }\\
\vspace {.5cm}

{Signed \_\_\_\_\_\_\_\_\_\_\_\_\_\_\_\_\_\_\_\_\_\_\_\_\_\_\_\_\_\_\_\_\_\_}


\today
\end{center}
\end{titlepage}



%%%%%%%%%%%%%%%%%%%%%%%%%%%%%%%%%%%%%%%%%%%%%%%%%%%%%%%%%%%
\author{George C. Micros}
\title{Written Assignment 8}
\subtitle{Fall 2014 \newline  CS 495: Introduction to Web Science\newline Dr. Michael Nelson}
\maketitle

%%%%%%%%%%%%%%%%%%%%%%%%%%%%%%%%%%%%%%%%%%%%%%%%%%%%%%%%%%%

\tableofcontents

\chapter{Written Assignment 8}

\noindent


The goal of this project is to use the basic recommendation principles
we have learned for user-collected data. You will modify the code
given to you which performs movie recommendations from the MovieLense
data sets.The MovieLense data sets were collected by the GroupLens Research
Project at the University of Minnesota during the seven-month period
from September 19th, 1997 through April 22nd, 1998. It is available
for download from \url{http://www.grouplens.org/node/73}

There are three files which we will use:


\begin{itemize}

\item  u.data: 100,000 ratings by 943 users on 1,682 movies. Each
user has rated at least 20 movies. Users and items are numbered
consecutively from 1. The data is randomly ordered. This is a tab
separated list of 

user id | item id | rating | timestamp

The time stamps are unix seconds since 1/1/1970 UTC.

Example:

196	242	3	881250949

186	302	3 	891717742

22	377	1 	878887116

244	51	2 	880606923

166	346	1 	886397596

298	474	4 	884182806

115	265	2	881171488

\item  u.item: Information about the 1,682 movies. This is a tab separated list of

movie id | movie title | release date | video release date | IMDb URL | unknown | Action | Adventure | Animation |Children's | Comedy | Crime | Documentary | Drama | Fantasy | Film-Noir | Horror | Musical | Mystery | Romance | Sci-Fi | Thriller | War | Western |

The last 19 fields are the genres, a 1 indicates the movie is of that genre, a 0 indicates it is not; movies can be in several genres at once. The movie ids are the ones used in the u.data data set.

Example:

161|Top Gun (1986)|01-Jan-1986||http://us.imdb.com/M/title-exact?Top%20Gun%20(1986)|0|1|0|0|0|0|0|0|0|0|0|0|0|0|1|0|0|0|0 
162|On Golden Pond (1981)|01-Jan-1981||http://us.imdb.com/M/title-exact?On%20Golden%20Pond%20(1981)|0|0|0|0|0|0|0|0|1|0|0|0|0|0|0|0|0|0|0 
163|Return of the Pink Panther, The (1974)|01-Jan-1974||http://us.imdb.com/M/title-exact?Return%20of%20the%20Pink%20Panther,%20The%20(1974)|0|0|0|0|0|1|0|0|0|0|0|0|0|0| 0|0|0|0|0

\item   u.user: Demographic information about the users. This is a tab
separated list of:

user id | age | gender | occupation | zip code

The user ids are the ones used in the u.data data set.

Example:

1|24|M|technician|85711 
2|53|F|other|94043 
3|23|M|writer|32067 
4|24|M|technician|43537 
5|33|F|other|15213

\end{itemize}

The code for reading from the u.data and u.item files and creating
recommendations is described in the book Programming Collective
Intelligence (check email for more details). You are to modify
recommendations.py to answer the following questions. Each question your
program answers correctly will award you 1 point.



The data was organized using classes for the movies, users and reviews. This helped organized the data in a way that was similar to the way it was represented in the files.

\begin{lstlisting}[caption={Class definitions for movies, users and reviews}]
class movie:
	def __init__(self, data):
		data = data.strip('\n').split('|')		
		self.id 	= int(data[0])
		self.title 	= data[1].replace(',', '')
		self.data 	= data[2]
		self.vid 	= data[3]
		self.url 	= data[4]
		self.genre 	= data[5:len(data)]
		self.scores	= []
		self.cnt	= 0
		self.avg 	= 0
		self.rvwrs	= []

	def avgr(self):
		if self.cnt > 0:
			self.avg = sum(self.scores)/self.cnt
		else:
			self.avg = 0

class user:
	def __init__(self,data):
		data = data.strip('\n').split('|')
		self.id 	= int(data[0])
		self.age	= int(data[1])
		self.sex	= data[2]
		self.job	= data[3]
		self.zip 	= data[4]
		self.film   = {}
		self.cnt	= 0

class review:
	def __init__(self, data):
		data = data.strip('\n').split('\t')
		self.user 	= int(data[0])
		self.item	= int(data[1])
		self.score	= int(data[2])
		self.time	= int(data[3])
\end{lstlisting}

Data was loaded from its respective files, parced and initialized into a new object that was appened to and list of similar objects.

\begin{lstlisting}[caption={Loading data from files to list of objects}]
f = open("./ml-100k/u.item")
for i in f:
	movies.append(movie(i))

f = open("./ml-100k/u.user")
for i in f:
	users.append(user(i))

f = open("./ml-100k/u.data")	
for i in f:
	reviews.append(review(i))
\end{lstlisting}



\section{Question 1}

\subsection{The Question}

\begin{flushleft}

Using D3, create a graph of the Karate club before and after
the split.

- Weight the edges with the data from: 
\url{http://vlado.fmf.uni-lj.si/pub/networks/data/ucinet/zachary.dat}

- Have the transition from before/after the split occur on a mouse
click.

\end{flushleft}
\subsection{The Answer}

\url{http://www.cs.odu.edu/~gmicros/karate/}



%
%\lstset{
%    language=R,
%    label=code:q1,
%    caption={R script to do the maths and plot}
%}
%\lstinputlisting{../q1/maths.r}
%%
%\lstset{
%    language=bash,
%    label=code:q1_test,
%    caption={Bash script to extract text from webpage, ignoring links}
%}
%\lstinputlisting{../process.sh}







\section{Question 2}

\subsection{The Question}

\begin{flushleft}

What 5 movies received the most ratings? Show the movies and
the number of ratings sorted by number of ratings.

\end{flushleft}
\subsection{The Answer}

There review infromation is read in and appeneded to each movie. There is a counter that counter the number of reviews for each movie. This could also be determined by checking the length of the review list for each movie. 


\begin{lstlisting}[caption={Python code for question 2}]

def q2(movies):
	clearScore(movies)
	for i in reviews:
		movies[i.item-1].scores.append(float(i.score))
		movies[i.item-1].cnt +=1

	for i in movies:
		i.avgr();

	# MOST RATING
	mRate = sorted(movies, key=lambda x:x.cnt, reverse=True)
	f = open("q2.txt", "w")
	print "\n\tQ2: Most Ratings"
	for i in range(0,5):
		print  mRate[i].cnt, mRate[i].title
		f.write("% d,  \"%s\"\n" % (mRate[i].cnt, mRate[i].title))
	f.close()

\end{lstlisting}

\begin{flushleft}

\begin{table}[h]
\centering
\setlength{\tabcolsep}{12pt}
\begin{tabular}{|ll|}

\hline
583 & Star Wars (1977)          \\ \hline
509 & Contact (1997)            \\ \hline
508 & Fargo (1996)              \\ \hline
507 & Return of the Jedi (1983) \\ \hline
485 & Liar Liar (1997)         \\ \hline
\end{tabular}
\caption{Movies with Most Ratings}
\end{table}
\end{flushleft}





\section{Question 3}

\subsection{The Question}

\begin{flushleft}

What 5 movies were rated the highest on average by women? Show
the movies and their ratings sorted by ratings.

\end{flushleft}
\subsection{The Answer}


\begin{flushleft}

\begin{table}[h]
\centering
\begin{tabular}{ll}
5.0000 & Year of the Horse (1997)                                \\
5.0000 & Visitors The (Visiteurs Les) (1993)                     \\
5.0000 & Telling Lies in America (1997)                          \\
5.0000 & Stripes (1981)                                          \\
5.0000 & Someone Else's America (1995)                           \\
5.0000 & Prefontaine (1997)                                      \\
5.0000 & Mina Tannenbaum (1994)                                  \\
5.0000 & Maya Lin: A Strong Clear Vision (1994)                  \\
5.0000 & Foreign Correspondent (1940)                            \\
5.0000 & Faster Pussycat! Kill! Kill! (1965)                     \\
5.0000 & Everest (1998)                                          \\ \hline
4.6329 & Schindler's List (1993)                                 \\ \hline
4.6316 & Close Shave A (1995)                                    \\ \hline 
4.5625 & Shawshank Redemption The (1994)                         \\ \hline
4.5333 & Wallace \& Gromit: The Best of Aardman Animation (1996) \\
   
\end{tabular}
\caption{Highest Average Rating by Women}
\end{table}


\end{flushleft}





\section{Question 4}

\subsection{The Question}

\begin{flushleft}

What 5 movies were rated the highest on average by men? Show
the movies and their ratings sorted by ratings.

\end{flushleft}
\subsection{The Answer}
\begin{flushleft}

The reviews were read line by line and before hey were appeneded to the respective movie there reviewer was checked to determine if he met the criterion of being a male. 


\begin{lstlisting}[caption={Python code for question 4}]
def q4(movies, reviews, users):
	clearScore(movies)
	for i in reviews:
		if users[i.user-1].sex == "M":
			movies[i.item-1].scores.append(float(i.score))
			movies[i.item-1].cnt +=1
	for i in movies:
		i.avgr(); 
	men = sorted(movies, key=lambda x:x.avg, reverse=True)
	f = open("q4.txt", "w")
	print "\n\tQ4: Highest Average by men"
	for i in range(0,30):
		print  '%.4f'%men[i].avg, men[i].title
		f.write("%.4f,  \"%s\"\n" % (men[i].avg, men[i].title))
	f.close()
\end{lstlisting}


\begin{table}[h]
\setlength{\tabcolsep}{12pt}
\centering
\begin{tabular}{|ll|}
\hline
5.0000 & Great Day in Harlem A (1994)                            \\
5.0000 & They Made Me a Criminal (1939)                          \\
5.0000 & Quiet Room The (1996)                                   \\
5.0000 & Hugo Pool (1997)                                        \\
5.0000 & Prefontaine (1997)                                      \\
5.0000 & Letter From Death Row A (1998)                          \\
5.0000 & Marlene Dietrich: Shadow and Light (1996)               \\
5.0000 & Star Kid (1997)                                         \\
5.0000 & Delta of Venus (1994)                                   \\
5.0000 & Saint of Fort Washington The (1993)                     \\
5.0000 & Santa with Muscles (1996)                               \\
5.0000 & Aiqing wansui (1994)                                    \\
5.0000 & Love Serenade (1996)                                    \\
5.0000 & Leading Man The (1996)                                  \\
5.0000 & Entertaining Angels: The Dorothy Day Story (1996)       \\
5.0000 & Little City (1998)                                      \\ \hline
4.6667 & Two or Three Things I Know About Her (1966)             \\
4.6667 & Little Princess The (1939)                              \\ \hline
4.6250 & Pather Panchali (1955)                                  \\ \hline
4.5000 & A Chef in Love (1996)                                   \\  
4.5000 & Anna (1996)                                             \\
4.5000 & Sliding Doors (1998)                                    \\
4.5000 & Grosse Fatigue (1994)                                   \\
4.5000 & Bitter Sugar (Azucar Amargo) (1996)                     \\ \hline
4.4734 & Casablanca (1942)                                       \\    \hline

\end{tabular}
\caption{Highest Average Rating by Men}
\end{table}




\end{flushleft}





\section{Question 4}

\subsection{The Question}

\begin{flushleft}

 Use MDS to create a JPEG of the blogs similar to slide 29.  
How many iterations were required?

\end{flushleft}
\subsection{The Answer}


The Python function used to produce the MDS plot is provided in the book \cite{PCI}. The algorithm required 353 iterations to converge. The terminal output it attached.

\lstset{
    language=Python,
    label=code:q1,
    caption={Python script to produce dendrograms}
}
\lstinputlisting{../q4/q4.py}


\begin{figure}
\centering
\includegraphics[width=\textwidth]{../q4/mds.png}
\caption{Terminal output from Multi-Dimensional Scaling}
\end{figure}

\begin{figure}
\centering
\includegraphics[width=\textwidth]{../q4/blogs2d}
\caption{Multi-Dimensional Scaling of the blogs}
\end{figure}





\section{Extra Credit Question}

\subsection{The Question}

\begin{flushleft}

Re-run question 2, but this time with proper TFIDF calculations
instead of the hack discussed on slide 7 (p. 32).  Use the same 500
words, but this time replace their frequency count with TFIDF scores
as computed in assignment \#3.  Document the code, techniques,
methods, etc. used to generate these TFIDF values.  Upload the new
data file to github.

Compare and contrast the resulting dendrogram with the dendrogram
from question \#2.

Note: ideally you would not reuse the same 500 terms and instead
come up with TFIDF scores for all the terms and then choose the top
500 from that list, but I'm trying to limit the amount of work
necessary.

\end{flushleft}
\subsection{The Answer}


The TFIDF calculation requires the computation of two numbers, the term frequency and the inverse document frequency.
The {\sl term frequency} (TF) is defined as the frequency of the term in the document normalized by the number of terms in the document. 

\begin{center}
\Large
$TF = \frac{\text{number of term occurances}}{\text{number of words in document}}$
\end{center}

The {\sl inverse document frequency} (IDF) of the query is defined as the log base 2 of the number of webpages in the corpus over the number of webpages containing the term. 




\begin{center}
\Large
$IDF =\log_{2} (\frac{\text{number of webpages in corpus}}{\text{number of webpages containing the term}})$
\end{center}


	In the context of the blog matrix this can be visualized easily. Each row consists of a blog vector and each column consists of a term column vector. The elements of the matrix are the occurance frequency of a given term in a specific blog. The code essentially scaled the frequency by the lenght of the blog feed, but that is constant across all terms. The term frequency of a term in a blog is the frequency divided by the number of terms in the blog. In the matrix that is the element value divided by the length of the blog vector. Then summing across all blogs gives the sum of the term frequencies of all blogs which results in a vector of length equal to the number of terms in the corpus. This is the TF value. 	The numerator of the IDF value is constant across all terms and therefore irrelevant. The denominator is the number of blogs that the term appeared it. In matrix view, this represents the number of non-zero elements in a column.
Unfortunately the implementation the book used was not matrix based and therefore some magic had to take place to figure out how to compute these values. However the proper use of dictionaries made the implemention efficient and straight-forward.




\begin{lstlisting}
tf = {}
    blogs = wordcounts
    # go through every blog
    for blog in blogs:
        # go through every term
        for term in blogs[blog]:
            if term in tf: 
                tf[term] += blogs[blog][term]/float(len(blogs[blog])) 
            else:
                tf[term] = blogs[blog][term]/float(len(blogs[blog])) 
    corp = float(len(blogs))
    idf = {}
    for i in apcount:
        idf[i] = log(corp/apcount[i])/log(2)
    tfidf = {}
    for i in tf:
        tfidf[i] = tf[i]*idf[i]
    new = []
    for i in tfidf:
        new.append((i, tfidf[i]))
    new.sort(key=lambda tup:tup[1], reverse=True)
    wordlist = []
    for i in range(500):
        print new[i][0]
        wordlist.append(new[i][0].encode('ascii'))

\end{lstlisting}

The overall accuracy of the two methods is hard to evaluate in an absolute scence because the blogs are not labeled in a way that provides insight to an absolute form of clustering. An attempt to create clusters manually based on the content is not objective and can be easily mislead by our perception of the blogs. Therefore there is no method to score the outcome can then compare the score of the two techniques. 

The one things that does seem to stand out is the distinct tree structure that the two methods display. This become more clear when they are placed alongside each other. The naive term frequency tree seems to randomly split at every iteration. Groupings split up in unequal groups at each step, i.e. a cluster will split up into two clusters, one containing a single blog and then other the rest. This shows that the groups are not very consistent and at the next split the outliers branch off into their own group. This process continues as each cluster loses a blog at each iteration and gets smaller, without any indication of a dictinct seperation into two groups. For example, the food cluster should split up into the desert and the entree clusters, but in this case the food cluster sheds a blog at a time. 

The TFIDF dendrogram appears more structured and precise, very close to binary tree. Throughout the entire process the clusters are divided into subgroups. There are occurances of the random ``one-off'' blog, but that is predictable. Also, the number of terms used is only 500, which many not be large enough to provide a proper result.  The TFIDF method appear to find dominant characteristics of the cluster to patition on. While the naive method finds the most eccentric member and removes him. 

One a more abstract, and slightly ``hand-wavy'', interpretation blogs represent the people that write them. A population can be clusters into in a variety of ways, i.e. hobby, occupation, gender, age, but the clusters that provide the most information are the ones that partition the cluster into the subgroups that are bound by a characteristic. This type of clustering is equivalent to a useful demographic map, e.g. young male engineers, young female engineers, retired wood craftsman, etc. The group can be used to describe its members in a reasonable way. However, the population can be clustered based on who is a Harvard graduate that is also in an underground alt-rock band and not.  There is lots of detail about one group, but there members of the other group are not held together by any common traits.  Adolf Hitler and Edward Snowden would be in the same cluster and then the question become what do they have in common. 
\begin{figure*}[t!]
        \centering
        \begin{subfigure}{0.5\textwidth}
	\centering
	 \includegraphics[height=11cm, width=9cm]{../q2/blogdendro.jpg}

             \caption{Dirty Hack}
        \end{subfigure}%
        \begin{subfigure}{0.5\textwidth}
		\centering
               \includegraphics[height=11cm, width=9cm]{../qec/blogdendro}	
\caption{TFIDF}
        \end{subfigure}
    
\end{figure*}



In conclusion, my opinion is that the TFIDF measure clusters in a way that produces distinct subgroups which are meaningful and consistent. On the other hand, the naive approach partitions based on the odd one out, but sometime the odd one isnt very odd and the non-odd ones dont have much to do with each other. This is the conclusion that I draw based on the structure of the divisions. Even if the groups don't seem related based on the title they have some common characteristic that TFIDF can determine, but the naive method does not.








\begin{figure}[h!]
\centering
\includegraphics[height=25cm]{../qec/blogdendro}
\caption{TFIDF dendrogram}
\end{figure}







\section{Question 7}

\subsection{The Question}

\begin{flushleft}

Which 5 raters most agreed with each other? Show the raters'
IDs and Pearson's r, sorted by r.


\end{flushleft}
\subsection{The Answer}


\begin{flushleft}
In order to determine the cluster of 5 most agreed reviewers all permutations of reviewers would have to be considered and some metric of correlation amongst them be computed. For the set of all permutations we wuld use k choose n, where k = 5 and n = 943

\begin{center}
$ \binom nk = \frac{n!}{k!\,(n-k)!} $ 

${943 \choose 5}  =  \frac{943!}{5!\,(9435)!}  =  6.148432630803 × 10^{12}$
\end{center}

which is really big and very impractical to have to go through that many iterations. 

The alternative is to use a greedy algorithm that is assumed to converge near the local minimum of this problem. By computing the nearest revierwer for each reviewer we can reduce the number of operations and produce a relationship with each reviewer. 

\begin{center}
p3,p9 

p9,p23 

p23,p25 

p25,p33 

p33,p77 
\end{center}

Once these relationship have been computed a chain of five uniques reviewers is produced. Many of these chains are degenerate, pointing to the same reviewers and repeating. 

\begin{center}
p3,p9 

p9,p3

p3,p9 

p9,p3 

p3,p9 
\end{center}


These can be eliminated easily. From the remaining chains a metric must be devised based on the weighed connections to determine the most agreed 5. The metric chosen was the sum of the correlations of every member to the remaining memebers. The higher the sum the most closely correlated the memebrs are to each other. The group with the highest sum was chosen and the correlation between users shown to be 1. 





\begin{lstlisting}[caption={Python code for question 7}]
def q7():
    prefs =  loadMovieLens(path='./ml-100k')   
    c = greedy(prefs, True)
    x = maths(c, prefs)

    x.sort(key=lambda x:x[5], reverse=True)
    y = []
    for i in x:
        if len(i) == len(set(i)):
            y.append(i)

    print "Most Agreed Reviewers"
    for i in range(0,5):
        print y[i]
\end{lstlisting}

\begin{lstlisting}[caption={Greddy Algorithm for determining most correlated reviewer}]
def greedy(prefs, best):
    c = [[]]*(len(prefs))
    for i in prefs:
        e = []
        for j in prefs:
            d = []
            num = sim_pearson(prefs,i,j);
            d.append(num)
            d.append(i)
            d.append(j)
            if (i != j):
                e.append(d)
        e.sort(key=lambda x:x[:][0], reverse=best)
        c[int(e[0][1])-1] = e[0]       
    return c 
\end{lstlisting}


\begin{lstlisting}[caption={Determining chain of 5 reviewers from correlated}]
def maths(c, prefs):
    d = []
    for i in c:
        e = []
        e.append(i[1])
        e.append(i[2])
        for j in range(2,5):
            e.append(c[int(e[j-1])-1][2])
        tSum = 0
        for j in range (0,5):
            for k in range (0,5):
                tSum +=  sim_pearson(prefs, e[j], e[k])
        e.append(tSum)
        d.append(e)
    return d
\end{lstlisting}


\begin{lstlisting}[caption={Python code for question 7}]
def q7():
    prefs =  loadMovieLens(path='./ml-100k')   
    c = greedy(prefs, True)
    x = maths(c, prefs)

    x.sort(key=lambda x:x[5], reverse=True)
    y = []
    for i in x:
        if len(i) == len(set(i)):
            y.append(i)

    print "Most Agreed Reviewers"
    for i in range(0,5):
        print y[i]
\end{lstlisting}

\begin{lstlisting}[caption={Correlation between two reviewers}]
def sim_pearson(prefs, p1, p2):
    '''
    Returns the Pearson correlation coefficient for p1 and p2.
    '''

    # Get the list of mutually rated items
    si = {}
    for item in prefs[p1]:
        if item in prefs[p2]:
            si[item] = 1
    # If they are no ratings in common, return 0
    if len(si) == 0:
        return 0
    # Sum calculations
    n = len(si)
    # Sums of all the preferences
    sum1 = sum([prefs[p1][it] for it in si])
    sum2 = sum([prefs[p2][it] for it in si])
    # Sums of the squares
    sum1Sq = sum([pow(prefs[p1][it], 2) for it in si])
    sum2Sq = sum([pow(prefs[p2][it], 2) for it in si])
    # Sum of the products
    pSum = sum([prefs[p1][it] * prefs[p2][it] for it in si])
    # Calculate r (Pearson score)
    num = pSum - sum1 * sum2 / n
    den = sqrt((sum1Sq - pow(sum1, 2) / n) * (sum2Sq - pow(sum2, 2) / n))
    if den == 0:
        return 0
    r = num / den
    return r
\end{lstlisting}

\begin{center}
Most Agreed Reviewers


$656 -> 909$ r =  1.00

$909 -> 869 $ r = 1.00

$869 -> 857$ r = 1.00

$857 -> 748$ r = 1.00

\vspace{10pt}

Highly Agreed Groups

['656', '909', '869', '857', '748', 21.743243005407813]

['520', '694', '306', '188', '732', 20.35143510753761]

['806', '909', '869', '857', '748', 20.28283545995625]

['249', '909', '869', '857', '748', 19.98780466540942]

['251', '909', '869', '857', '748', 19.714669425731906]



\end{center}


\end{flushleft}





\section{Question 8}

\subsection{The Question}

\begin{flushleft}

Which 5 raters most disagreed with each other (negative
correlation)? Show the raters' IDs and Pearson's r, sorted by r.


\end{flushleft}
\subsection{The Answer}


\begin{flushleft}


The solution for this question is similar to that of question 7. However the main different being that the goal is to find the users that disagree the most with each other, being least correlated. The main difference in the coe implementing this solution is the order of sorting. The sorting was reverese from question 7, so that the least correlated naighbors are chosen from the greedy algorithm. The least correlated group was chosen using the same metric as previously.




\begin{lstlisting}[caption={Python code for question 8}]
def q8():
    prefs =  loadMovieLens(path='./ml-100k')
    c = greedy(prefs, False)
    x = maths(c, prefs)

    x.sort(key=lambda x:x[5], reverse=False)
    y = []
    for i in x:
        if len(i) == len(set(i)):
            y.append(i)
    print "Least Agreed Reviewers"
    for i in range(0,5):
        print y[i]
\end{lstlisting}

\begin{lstlisting}[caption={Greddy Algorithm for determining most correlated reviewer}]
def greedy(prefs, best):
    c = [[]]*(len(prefs))
    for i in prefs:
        e = []
        for j in prefs:
            d = []
            num = sim_pearson(prefs,i,j);
            d.append(num)
            d.append(i)
            d.append(j)
            if (i != j):
                e.append(d)
        e.sort(key=lambda x:x[:][0], reverse=best)
        c[int(e[0][1])-1] = e[0]       
    return c 
\end{lstlisting}


\begin{lstlisting}[caption={Determining chain of 5 reviewers from correlated}]
def maths(c, prefs):
    d = []
    for i in c:
        e = []
        e.append(i[1])
        e.append(i[2])
        for j in range(2,5):
            e.append(c[int(e[j-1])-1][2])
        tSum = 0
        for j in range (0,5):
            for k in range (0,5):
                tSum +=  sim_pearson(prefs, e[j], e[k])
        e.append(tSum)
        d.append(e)
    return d
\end{lstlisting}


\begin{lstlisting}[caption={Correlation between two reviewers}]
def sim_pearson(prefs, p1, p2):
    '''
    Returns the Pearson correlation coefficient for p1 and p2.
    '''

    # Get the list of mutually rated items
    si = {}
    for item in prefs[p1]:
        if item in prefs[p2]:
            si[item] = 1
    # If they are no ratings in common, return 0
    if len(si) == 0:
        return 0
    # Sum calculations
    n = len(si)
    # Sums of all the preferences
    sum1 = sum([prefs[p1][it] for it in si])
    sum2 = sum([prefs[p2][it] for it in si])
    # Sums of the squares
    sum1Sq = sum([pow(prefs[p1][it], 2) for it in si])
    sum2Sq = sum([pow(prefs[p2][it], 2) for it in si])
    # Sum of the products
    pSum = sum([prefs[p1][it] * prefs[p2][it] for it in si])
    # Calculate r (Pearson score)
    num = pSum - sum1 * sum2 / n
    den = sqrt((sum1Sq - pow(sum1, 2) / n) * (sum2Sq - pow(sum2, 2) / n))
    if den == 0:
        return 0
    r = num / den
    return r
\end{lstlisting}

\begin{center}
Least Agreed Reviewers

$853 -> 336$ r =  -1.00

$336 -> 414$ r = -1.00

$414 -> 641$ r = -1.00

$641 -> 134$ r = -1.00

\vspace{10pt}

Highly Disagreed Groups

['853', '336', '414', '641', '134', -8.178267558914557]

['359', '760', '928', '432', '180', -5.94077103218698]

['895', '760', '928', '432', '180', -5.847371703822146]

['425', '477', '811', '441', '475', -5.381056081661583]

['67', '698', '898', '599', '19', -5.246363309059925]

\end{center}

\end{flushleft}





\section{Question 9}

\subsection{The Question}

\begin{flushleft}

What movie was rated highest on average by men over 40? By men
under 40?

\end{flushleft}
\subsection{The Answer}
\begin{flushleft}
This quesion is similar to question 4 where the highst rated movies by men were requested. This question adds the additional condition for age of the reviewer. The reviews were appened to each movie based on the conditions and then avergaed. 


\begin{lstlisting}[caption={Python code to for men over 40}]
def q9a(movies, reviews, users):
	clearScore(movies)
	for i in reviews:
		if (users[i.user-1].sex == "M") and (users[i.user-1].age > 40):
			movies[i.item-1].scores.append(float(i.score))
			movies[i.item-1].cnt +=1
	for i in movies:
		i.avgr(); 
	m40 = sorted(movies,key=lambda x:(x.avg, x.title), reverse=True)
	f = open("q9a.txt", "w")
	print "\n\tQ9a: Highest Average by men over 40"
	for i in range(0,30):
		print  '%.4f'%m40[i].avg, m40[i].title
		f.write("%.4f,  \"%s\"\n" % (m40[i].avg, m40[i].title))
	f.close()
\end{lstlisting}


\begin{lstlisting}[caption={Python code to for men under 40}]
def q9b(movies, reviews, users):
	clearScore(movies)
	for i in reviews:
		if (users[i.user-1].sex == "M") and (users[i.user-1].age < 40):
			movies[i.item-1].scores.append(float(i.score))
			movies[i.item-1].cnt +=1
	for i in movies:
		i.avgr(); 
	m30 = sorted(movies,key=lambda x:(x.avg, x.title), reverse=True)
	f = open("q9b.txt", "w")
	print "\n\tQ9b: Highest Average by men under 40"
	for i in range(0,40):
		print  '%.4f'%m30[i].avg, m30[i].title 	
		f.write("%.4f,  \"%s\"\n" % (m30[i].avg, m30[i].title ))
	f.close()
\end{lstlisting}

\begin{table}[h]
\centering
\setlength{\tabcolsep}{12pt}
\begin{tabular}{|ll|}
\hline
5.0000 & World of Apu The (Apur Sansar) (1959)                 \\
5.0000 & Unstrung Heroes (1995)                                \\
5.0000 & Two or Three Things I Know About Her (1966)           \\
5.0000 & They Made Me a Criminal (1939)                        \\
5.0000 & Strawberry and Chocolate (Fresa y chocolate) (1993)   \\
5.0000 & Star Kid (1997)                                       \\
5.0000 & Spice World (1997)                                    \\
5.0000 & Solo (1996)                                           \\
5.0000 & Rendezvous in Paris (Rendez-vous de Paris Les) (1995) \\
5.0000 & Prefontaine (1997)                                    \\
5.0000 & Poison Ivy II (1995)                                  \\
5.0000 & Marlene Dietrich: Shadow and Light (1996)             \\
5.0000 & Little Princess The (1939)                            \\
5.0000 & Little City (1998)                                    \\
5.0000 & Leading Man The (1996)                                \\
5.0000 & Late Bloomers (1996)                                  \\
5.0000 & Indian Summer (1996)                                  \\
5.0000 & Hearts and Minds (1996)                               \\
5.0000 & Great Day in Harlem A (1994)                          \\
5.0000 & Grateful Dead (1995)                                  \\
5.0000 & Faithful (1996)                                       \\
5.0000 & Double Happiness (1994)                               \\
5.0000 & Boxing Helena (1993)                                  \\
5.0000 & Aparajito (1956)                                      \\
5.0000 & Ace Ventura: When Nature Calls (1995)                 \\ \hline
4.8000 & Pather Panchali (1955)                                \\ \hline
4.6667 & Whole Wide World The (1996)                           \\
4.6667 & A Chef in Love (1996)                                 \\ \hline
4.6471 & Close Shave A (1995)                                  \\ \hline
4.6000 & Shanghai Triad (Yao a yao yao dao waipo qiao) (1995)  \\ \hline
\end{tabular}
\caption{Highest Average Rating by Men over 40}
\end{table}

\begin{table}[h]
\centering
\setlength{\tabcolsep}{12pt}
\begin{tabular}{|ll|} \hline
5.0000 & Star Kid (1997)                                         \\ 
5.0000 & Santa with Muscles (1996)                               \\
5.0000 & Saint of Fort Washington The (1993)                     \\
5.0000 & Quiet Room The (1996)                                   \\
5.0000 & Prefontaine (1997)                                      \\
5.0000 & Perfect Candidate A (1996)                              \\
5.0000 & Maya Lin: A Strong Clear Vision (1994)                  \\
5.0000 & Magic Hour The (1998)                                   \\
5.0000 & Love in the Afternoon (1957)                            \\
5.0000 & Love Serenade (1996)                                    \\
5.0000 & Letter From Death Row A (1998)                          \\
5.0000 & Leading Man The (1996)                                  \\
5.0000 & Hugo Pool (1997)                                        \\
5.0000 & Entertaining Angels: The Dorothy Day Story (1996)       \\
5.0000 & Delta of Venus (1994)                                   \\
5.0000 & Crossfire (1947)                                        \\
5.0000 & Angel Baby (1995)                                       \\
5.0000 & Aiqing wansui (1994)                                    \\ \hline
4.5000 & Winter Guest The (1997)                                 \\
4.5000 & Two or Three Things I Know About Her (1966)             \\
4.5000 & Sum of Us The (1994)                                    \\
4.5000 & Sliding Doors (1998)                                    \\
4.5000 & Man of No Importance A (1994)                           \\
4.5000 & Little Princess The (1939)                              \\
4.5000 & Innocents The (1961)                                    \\
4.5000 & Grosse Fatigue (1994)                                   \\
4.5000 & Fille seule La (A Single Girl) (1995)                   \\
4.5000 & Boy's Life 2 (1997)                                     \\
4.5000 & Anna (1996)                                             \\  \hline
4.4762 & Wallace \& Gromit: The Best of Aardman Animation (1996) \\ \hline
4.4754 & Casablanca (1942)                                       \\ \hline
4.4706 & Paths of Glory (1957)                                   \\ \hline

\end{tabular}
\caption{Highest Average Rating by Men under 40}
\end{table}


\end{flushleft}





\section{Question 10}

\subsection{The Question}

\begin{flushleft}

What movie was rated highest on average by women over 40? By
women under 40?

\end{flushleft}
\subsection{The Answer}


\begin{flushleft}

This quesion is similar to question 3 where the highst rated movies by women were requested. This question adds the additional condition for age of the reviewer. The reviews were appened to each movie based on the conditions and then avergaed. 


\begin{lstlisting}[caption={Python code to for women over 40}]
def q10a(movies, reviews, users):
	clearScore(movies)
	for i in reviews:
		if (users[i.user-1].sex == "F") and (users[i.user-1].age > 40):
			movies[i.item-1].scores.append(float(i.score))
			movies[i.item-1].cnt +=1
	for i in movies:
		i.avgr(); 
	w40 = sorted(movies,key=lambda x:(x.avg, x.title), reverse=True)
	f = open("q10a.txt", "w")
	print "\n\tQ10a: Highest Average by women over 40"
	for i in range(0,40):
		print  '%.4f'%w40[i].avg, w40[i].title
		f.write("%.4f,  \"%s\"\n" % (w40[i].avg, w40[i].title))
	f.close()
\end{lstlisting}


\begin{lstlisting}[caption={Python code to for women under 40}]
def q10b(rmovies, reviews, users):
	clearScore(movies)
	for i in reviews:
		if (users[i.user-1].sex == "F") and (users[i.user-1].age < 40):
			movies[i.item-1].scores.append(float(i.score))
			movies[i.item-1].cnt +=1
	for i in movies:
		i.avgr(); 
	w30 = sorted(movies,key=lambda x:(x.avg, x.title), reverse=True)
	f = open("q10b.txt", "w")
	print "\n\tQ10b: Highest Average by women under 40"
	for i in range(0,40):
		print  '%.4f'%w30[i].avg, w30[i].title	
		f.write("%.4f,  \"%s\"\n" % (w30[i].avg, w30[i].title))
	f.close()
\end{lstlisting}



\begin{table}[h]
\centering
\setlength{\tabcolsep}{12pt}
\begin{tabular}{|ll|}
\hline
5.0000 & Wrong Trousers The (1993)                        \\
5.0000 & Visitors The (Visiteurs Les) (1993)              \\
5.0000 & Top Hat (1935)                                   \\
5.0000 & Tombstone (1993)                                 \\
5.0000 & Swept from the Sea (1997)                        \\
5.0000 & Shallow Grave (1994)                             \\
5.0000 & Shall We Dance? (1937)                           \\
5.0000 & Safe (1995)                                      \\
5.0000 & Quest The (1996)                                 \\
5.0000 & Pocahontas (1995)                                \\
5.0000 & Nightmare Before Christmas The (1993)            \\
5.0000 & Mina Tannenbaum (1994)                           \\
5.0000 & Mary Shelley's Frankenstein (1994)               \\
5.0000 & Ma vie en rose (My Life in Pink) (1997)          \\
5.0000 & Letter From Death Row A (1998)                   \\
5.0000 & In the Bleak Midwinter (1995)                    \\
5.0000 & Great Dictator The (1940)                        \\
5.0000 & Grand Day Out A (1992)                           \\
5.0000 & Gold Diggers: The Secret of Bear Mountain (1995) \\
5.0000 & Funny Face (1957)                                \\
5.0000 & Foreign Correspondent (1940)                     \\
5.0000 & Bride of Frankenstein (1935)                     \\
5.0000 & Best Men (1997)                                  \\
5.0000 & Band Wagon The (1953)                            \\
5.0000 & Balto (1995)                                     \\
5.0000 & Angel Baby (1995)                                \\ \hline
4.8000 & Once Were Warriors (1994)                        \\ \hline
4.7000 & Fantasia (1940)                                  \\ \hline
4.6667 & Last of the Mohicans The (1992)                  \\ \hline
4.5714 & Christmas Carol A (1938)                         \\ \hline

\end{tabular}
\caption{Highest Average Rating by Women over 40}
\end{table}



\begin{table}[h]
\centering
\setlength{\tabcolsep}{12pt}
\begin{tabular}{|ll|}
\hline
5.0000 & Year of the Horse (1997)                                                        \\
5.0000 & Wedding Gift The (1994)                                                         \\
5.0000 & Umbrellas of Cherbourg The (Parapluies de Cherbourg Les) (1964)                 \\
5.0000 & Telling Lies in America (1997)                                                  \\
5.0000 & Stripes (1981)                                                                  \\
5.0000 & Someone Else's America (1995)                                                   \\
5.0000 & Prefontaine (1997)                                                              \\
5.0000 & Nico Icon (1995)                                                                \\
5.0000 & Mina Tannenbaum (1994)                                                          \\
5.0000 & Maya Lin: A Strong Clear Vision (1994)                                          \\
5.0000 & Horseman on the Roof The (Hussard sur le toit Le) (1995)                        \\
5.0000 & Heaven's Prisoners (1996)                                                       \\
5.0000 & Grace of My Heart (1996)                                                        \\
5.0000 & Faster Pussycat! Kill! Kill! (1965)                                             \\
5.0000 & Everest (1998)                                                                  \\
5.0000 & Don't Be a Menace to South Central While Drinking Your Juice in the Hood (1996) \\
5.0000 & Backbeat (1993)                                                                 \\ \hline
4.8182 & Wallace \& Gromit: The Best of Aardman Animation (1996)                         \\ \hline
4.8000 & Paradise Lost: The Child Murders at Robin Hood Hills (1996)                     \\ \hline
4.8000 & Anne Frank Remembered (1995)                                                    \\
4.7021 & Shawshank Redemption The (1994)                                                 \\ \hline
4.7000 & Shall We Dance? (1996)                                                          \\ \hline

\end{tabular}
\caption{Highest Average Rating by Women under 40}
\end{table}

\end{flushleft}





%%%%%%%%%%%%%%%%%%%%%%%%%%%%%%%%%%%%%%%%%%%%%%%%%%%%%%%%%%%


%%%%%%%%%%%%%%%%%%%%%%%%%%%%%%%%%%%%%%%%%%%%%%%%%%%%%%%%%%%
\nocite{*}
\bibliography{sources}
\bibliographystyle{unsrt}




%%%%%%%%%%%%%%%%%%%%%%%%%%%%%%%%%%%%%%%%%%%%%%%%%%%%%%%%%%%

%%%%%%%%%%%%%%%%%%%%%%%%%%%%%%%%%%%%%%%%%%%%%%%%%%%%%%%%%%%



%%%%%%%%%%%%%%%%%%%%%%%%%%%%%%%%%%%%%%%%%%%%%%%%%%%%%%%%%%%




%%%%%%%%%%%%%%%%%%%%%%%%%%%%%%%%%%%%%%%%%%%%%%%%%%%%%%%%%%%



%%%%%%%%%%%%%%%%%%%%%%%%%%%%%%%%%%%%%%%%%%%%%%%%%%%%%%%%%%%




\end{document}


